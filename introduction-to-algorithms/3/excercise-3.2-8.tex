\documentclass{article}
\usepackage{amsmath}
\begin{document}

    Prove that:
    \begin{align}
        k\cdot ln(n)=\theta(n)
    \end{align}

    implies:
    \begin{align}
        k = \theta(\frac{n}{ln(n)})
    \end{align}

    Let's write it in theta notation:
    \begin{align}
        c_1 \cdot n \leq k \cdot ln(k) \leq c_2 \cdot n
    \end{align}

    and
    \begin{align}
        c_1 \cdot \frac{n}{ln(n)} \leq k \leq c_2 \cdot \frac{n}{ln(n)}
    \end{align}

    considering both we can write:
    \begin{align}
          c_1 \cdot n \leq c_{2} \cdot \frac{n}{ln(n)} \cdot ln(\frac{n}{ln(n)})
    \end{align}

    next step:
    \begin{align}
          c \leq \frac{1}{ln(n)} \cdot (ln(n)-ln(ln(n)))
    \end{align}

    next step:
    \begin{align}
          c \leq 1 - \frac{ln(ln(n))}{ln(n)}
    \end{align}

    as we can see
    \begin{align}
          \lim_{x\to\infty} \frac{ln(ln(n))}{ln(n)} = 0
    \end{align}

    so the left part goes to 0 and we can find c to satisfy the equation:
    \begin{align}
          c = \left( 0,1 \right)
    \end{align}

    similarly for the right part of the equation:
     \begin{align}
          c = \left( 1, \infty)
    \end{align}


\end{document}
